\input{../utils/preamble}
\createdgmtitle{7}
%--------------------------------------------------------------------------------
\begin{document}
%--------------------------------------------------------------------------------
\begin{frame}[noframenumbering,plain]
%\thispagestyle{empty}
\titlepage
\end{frame}
%=======
\begin{frame}{Recap of previous lecture}
	\begin{block}{Gaussian autoregressive flow (MAF)}
		\vspace{-0.8cm}
		\begin{align*}
			\bx &= g(\bz, \btheta) \quad \Rightarrow \quad x_i = \sigma_i (\bx_{1:i-1}) \cdot z_i + \mu_i(\bx_{1:i-1}). \\
			\bz &= f(\bx, \btheta) \quad \Rightarrow \quad z_i = \left(x_i - \mu_i(\bx_{1:i-1}) \right) \cdot \frac{1}{ \sigma_i (\bx_{1:i-1})}.
		\end{align*}
		Generation function $g(\bz, \btheta)$ is \textbf{sequential}. Inference function $f(\bx, \btheta)$ is \textbf{not sequential}.
	\end{block}
	\begin{block}{Inverse autoregressive flow (IAF)}
		\vspace{-0.5cm}
		\begin{align*}
			\bx &= g(\bz, \btheta) \quad \Rightarrow \quad x_i = \tilde{\sigma}_i (\bz_{1:i-1}) \cdot z_i + \tilde{\mu}_i(\bz_{1:i-1}) \\
			\bz &= f(\bx, \btheta) \quad \Rightarrow \quad z_i = \left( x_i - \tilde{\mu}_i(\bz_{1:i-1})\right) \cdot \frac{1}{\tilde{\sigma}_i (\bz_{1:i-1}) }.
		\end{align*}
		\vspace{-0.3cm}
	\end{block}
		
	\myfootnote{
	\href{https://arxiv.org/abs/1705.07057}{Papamakarios G., Pavlakou T., Murray I. Masked Autoregressive Flow for Density Estimation, 2017} \\
	\href{https://arxiv.org/abs/1606.04934}{Kingma D. P. et al. Improving Variational Inference with Inverse Autoregressive Flow, 2016} } 
\end{frame}
%=======
\begin{frame}{Recap of previous lecture}
	\begin{block}{Autoregressive flows}
		\begin{figure}
			\includegraphics[width=0.9\linewidth]{figs/autoregressive_flow}
		\end{figure}
	\end{block}
	\begin{block}{RealNVP: Affine coupling law}
		\vspace{-0.7cm}
		\[
			\begin{cases} \bz_{1:d} = \bx_{1:d}; \\ \bz_{d:m} = \tau (\bx_{d:m}, c(\bx_{1:d}));\end{cases} 
			\quad \Leftrightarrow \quad 
			\begin{cases} \bx_{1:d} = \bz_{1:d}; \\ \bx_{d:m} = \tau^{-1} (\bz_{d:m}, c(\bz_{1:d})).\end{cases}
		\]
	\end{block}
	\myfootnote{
	\href{https://arxiv.org/abs/1410.8516}{Dinh L., Krueger D., Bengio Y. NICE: Non-linear Independent Components Estimation, 2014} \\
	\href{https://arxiv.org/abs/1605.08803}{Dinh L., Sohl-Dickstein J., Bengio S. Density estimation using Real NVP, 2016}}
\end{frame}
%=======
\begin{frame}{Outline}
	\tableofcontents
\end{frame}
\AtBeginSection[ ]
{
	\begin{frame}{Outline}
		\tableofcontents[currentsection]
	\end{frame}
}
%=======
\begin{frame}{Optimal VAE prior}
	How to choose the optimal $p(\bz)$?
	\begin{itemize}
		\item Standard Gaussian $p(\bz) = \mathcal{N}(0, I)$ $\Rightarrow$ over-regularization;
		\item $p(\bz) = q_{\text{agg}}(\bz) = \frac{1}{n}\sum_{i=1}^n q(\bz | \bx_i)$ $\Rightarrow$ overfitting and highly expensive.
	\end{itemize}
	\vspace{-0.3cm}
	\begin{minipage}[t]{0.5\columnwidth}
		\begin{block}{Non learnable prior $p(\bz)$}
			\begin{figure}[h]
				\centering
				\includegraphics[width=0.8\linewidth]{figs/non_learnable_prior}
			\end{figure}
		\end{block}
	\end{minipage}%
	\begin{minipage}[t]{0.5\columnwidth}
		\begin{block}{Learnable prior $p(\bz | \blambda)$}
			\begin{figure}[h]
				\centering
				\includegraphics[width=0.8\linewidth]{figs/learnable_prior}
			\end{figure}
		\end{block}
	\end{minipage}
	\myfootnotewithlink{https://jmtomczak.github.io/blog/7/7\_priors.html}{image credit: https://jmtomczak.github.io/blog/7/7\_priors.html}
\end{frame}
%=======
\begin{frame}{Learnable VAE prior}
	\begin{block}{Optimal prior}
		\vspace{-0.7cm}
		\[
			KL(q_{\text{agg}}(\bz) || p(\bz)) = 0 \quad \Leftrightarrow \quad p (\bz) = q_{\text{agg}}(\bz) = \frac{1}{n} \sum_{i=1}^n q(\bz | \bx_i).
		\]
		\vspace{-0.6cm}
	\end{block}
	\begin{block}{Mixture of Gaussians}
		\vspace{-0.3cm}
		\[
			p(\bz | \blambda) = \sum_{k=1}^K w_k \cN(\bz | \bmu_k, \bsigma_k^2), \quad \blambda = \{w_k, \bmu_k, \bsigma_k\}_{k=1}^K.
		\]
		\vspace{-0.5cm}
	\end{block}
	\begin{block}{Variational Mixture of posteriors (VampPrior)}
		\vspace{-0.3cm}
		\[
		p(\bz | \blambda) = \frac{1}{K} \sum_{k=1}^K q(\bz | \bu_k),
		\]
		where $\blambda = \{\bu_1, \dots, \bu_K\}$ are trainable pseudo-inputs.
	\end{block}
	\begin{itemize}
		\item Multimodal $\Rightarrow$ prevents over-regularization;.
		\item $K \ll n$ $\Rightarrow$ prevents from potential overfitting + less expensive to train.
	\end{itemize}
	\myfootnotewithlink{https://arxiv.org/abs/1705.07120}{Tomczak J. M., Welling M. VAE with a VampPrior, 2017}
\end{frame}
%=======
\begin{frame}{VampPrior}
	\begin{itemize}
	\item Do we really need the multimodal prior?
	\item Is it beneficial to couple the prior with the variational posterior or the MoG prior is enough?
	\end{itemize}
	\begin{block}{Results}
		\vspace{-0.3cm}
		\begin{minipage}[t]{0.55\columnwidth}
			\begin{figure}[h]
				\centering
				\includegraphics[width=1.0\linewidth]{figs/VampPrior_1.png}
			\end{figure}
		\end{minipage}%
		\begin{minipage}[t]{0.45\columnwidth}
			\begin{figure}[h]
				\centering
				\includegraphics[width=1.0\linewidth]{figs/VampPrior_2.png}
			\end{figure}
		\end{minipage}
	\end{block}
	\textbf{Top row:} generated images by PixelHVAE + VampPrior for chosen pseudo-input in the left top corner. \\
	\vspace{0.1cm}
	\textbf{Bottom row:} pseudo-inputs for different datasets.
	\myfootnotewithlink{https://arxiv.org/abs/1705.07120}{Tomczak J. M., Welling M. VAE with a VampPrior, 2017}
\end{frame}
%=======
\begin{frame}{Flows-based VAE prior}
	\begin{block}{Flow model in latent space}
		\vspace{-0.5cm}
		\[
			\log p(\bz | \blambda) = \log p(\bepsilon) + \log \det \left | \frac{d \bepsilon}{d\bz}\right| = \log p(\bepsilon) + \log \det \left | \frac{\partial f(\bz, \blambda)}{\partial \bz}\right| 
		\]
		\[
			\bz = g(\bepsilon, \blambda) = f^{-1}(\bepsilon, \blambda)
		\]
	\end{block}
	\vspace{-0.3cm}
	\begin{itemize}
		\item RealNVP flow.
		\item Autoregressive flow (MAF).
	\end{itemize}
	Why it is not a good idea to use IAF for VAE prior?
	\begin{block}{ELBO with flow-based VAE prior}
		\vspace{-0.5cm}
		{\footnotesize
		\begin{multline*}
			\mathcal{L}(\bphi, \btheta) = \mathbb{E}_{q(\bz | \bx, \bphi)} \left[ \log p(\bx | \bz, \btheta) +  \log p(\bz | \blambda) - \log q(\bz | \bx, \bphi) \right] \\
				= \mathbb{E}_{q(\bz | \bx, \bphi)} \left[ \log p(\bx | \bz, \btheta) + \underbrace{ \Bigl( \log p(f(\bz, \blambda)) + \log \left| \det \frac{\partial f(\bz, \blambda)}{\partial \bz} \right| \Bigr) }_{\text{flow-based prior}} - \log q(\bz | \bx, \bphi) \right] 
		\end{multline*}
		}
	\end{block}
	\myfootnotewithlink{https://arxiv.org/abs/1611.02731}{Chen X. et al. Variational Lossy Autoencoder, 2016}
\end{frame}
%=======
\begin{frame}{VAE limitations}
	\begin{itemize}
		\item Poor generative distribution (decoder)
		\[
			p(\bx | \bz, \btheta) = \mathcal{N}(\bx| \bmu_{\btheta}(\bz), \bsigma^2_{\btheta}(\bz)) \quad \text{or } = \text{Softmax}(\bpi_{\btheta}(\bz)).
		\]
		\item Loose lower bound
		\[
			\log p(\bx | \btheta) - \mathcal{L}(q, \btheta) = (?).
		\]
		\item Poor prior distribution
		\[
			p(\bz) = \mathcal{N}(0, \mathbf{I}).
		\]
		\item \textbf{Poor variational posterior distribution (encoder)}
		\[
			q(\bz | \bx, \bphi) = \mathcal{N}(\bz| \bmu_{\bphi}(\bx), \bsigma^2_{\bphi}(\bx)).
		\]
	\end{itemize}
\end{frame}
%=======
\begin{frame}{Variational posterior}
	\begin{block}{ELBO}
		\[
		\log p(\bx | \btheta) = \mathcal{L}(q, \btheta) + KL(q(\bz | \bx, \bphi) || p(\bz | \bx, \btheta)).
		\]
	\end{block}
	\begin{itemize}
		\item In E-step of EM-algorithm we wish $KL(q(\bz | \bx, \bphi) || p(\bz | \bx, \btheta)) = 0$. \\
		(In this case the lower bound is tight $\log p(\bx | \btheta) = \mathcal{L}(q, \btheta)$). \\
		\item Normal variational distribution $q(\bz | \bx, \bphi) = \mathcal{N}(\bz| \bmu_{\bphi}(\bx), \bsigma^2_{\bphi}(\bx))$ is poor (e.g. has only one mode). \\
		\item Flows models convert a simple base distribution to a complex one using invertible transformation with simple Jacobian. How to use flows in VAE posterior?
	\end{itemize}
\end{frame}
%=======
\begin{frame}{Flows in VAE posterior}
	Apply a sequence of transformations to the random variable
	\[
	\bz_0 \sim q(\bz | \bx, \bphi) = \mathcal{N}(\bz| \bmu_{\bphi}(\bx), \bsigma^2_{\bphi}(\bx)).
	\]
	Let $q(\bz | \bx, \bphi)$ (VAE encoder) be a base distribution for a flow model.
	
	\begin{block}{Flow model in latent space}
		\vspace{-0.3cm}
		\[
			\log q(\bz^* | \bx, \bphi, \blambda) = \log q(\bz | \bx, \bphi) + \log \det \left | \frac{\partial f(\bz, \blambda)}{\partial \bz}\right|
		\]
		\[
			\bz^* = f(\bz, \blambda) = g^{-1}(\bz, \blambda)
		\]
	\end{block}
	Here $f(\bz, \blambda)$ is a flow model (e.g. stack of planar/coupling layers) parameterized by $\blambda$.
	
	
	Let use $q(\bz^* | \bx, \bphi, \blambda)$ as a variational distribution. Here $\bphi$~-- encoder parameters, $\blambda$~-- flow parameters.
	
	\myfootnotewithlink{https://arxiv.org/abs/1505.05770}{Rezende D. J., Mohamed S. Variational Inference with Normalizing Flows, 2015} 
\end{frame}
%=======
\begin{frame}{Flows-based VAE posterior}
	\begin{itemize}
		\item Encoder outputs base distribution $q(\bz | \bx, \bphi)$.
		\item Flow model $\bz^* = f(\bz, \blambda)$ transforms the base distribution $q(\bz | \bx, \bphi)$ to the distribution $q(\bz^* | \bx, \bphi, \blambda)$.
		\item Distribution $q(\bz^* | \bx, \bphi, \blambda)$ is used as a variational distribution for ELBO maximization.
	\end{itemize}
	
	\begin{block}{Flow model in latent space}
		\vspace{-0.3cm}
		\[
			\log q(\bz^* | \bx, \bphi, \blambda) = \log q(\bz | \bx, \bphi) + \log \det \left | \frac{\partial f(\bz, \blambda)}{\partial \bz}\right|
		\]
		\vspace{-0.3cm}
	\end{block}
	\begin{block}{ELBO with flow-based VAE posterior}
		\vspace{-0.5cm}
		\begin{align*}
			\mathcal{L} (\bphi, \btheta, \blambda)  
			&= \bbE_{q(\bz^* | \bx, \bphi, \blambda)} \bigl[\log p(\bx, \bz^* | \btheta) - \log q(\bz^*| \bx, \bphi, \blambda) \bigr] \\ 
			&=  \bbE_{q(\bz^* | \bx, \bphi, \blambda)} \log p(\bx | \bz^*, \btheta) - KL (q(\bz^* | \bx, \bphi, \blambda) || p(\bz^*)).
		\end{align*}
		\vspace{-0.4cm}
	\end{block}
	The second term in ELBO is reverse KL divergence. Planar flows was originally proposed for variational inference in VAE.
	
	\myfootnotewithlink{https://arxiv.org/abs/1505.05770}{Rezende D. J., Mohamed S. Variational Inference with Normalizing Flows, 2015} 
\end{frame}
%=======
\begin{frame}{Flows-based VAE posterior}
	
	\begin{block}{Flow model in latent space}
		\vspace{-0.3cm}
		\[
			\log q(\bz^* | \bx, \bphi, \blambda) = \log q(\bz | \bx, \bphi) + \log \det \left | \frac{\partial f(\bz, \blambda)}{\partial \bz}\right|
		\]
		\vspace{-0.3cm}
	\end{block}
	\begin{block}{ELBO objective}
		\vspace{-0.5cm}
		\begin{multline*}
			\mathcal{L} (\bphi, \btheta, \blambda)  
			= \mathbb{E}_{q(\bz^* | \bx, \bphi, \blambda)} \bigl[\log p(\bx, \bz^* | \btheta) - \log q(\bz^*| \bx, \bphi, \blambda) \bigr] = \\
			= \mathbb{E}_{q(\bz | \bx, \bphi)} \left. \bigl[\log p(\bx, \bz^* | \btheta) - \log q(\bz^*| \bx, \bphi, \blambda) \bigr]\right|_{\bz^* = f(\bz, \blambda)} = \\
			= \mathbb{E}_{q(\bz | \bx, \bphi)} \bigg[\log p(\bx, \bz^* | \btheta) -  \log q(\bz | \bx, \bphi ) + \log \left| \det \left( \frac{\partial f(\bz, \blambda)}{\partial \bz} \right) \right| \bigg].
		\end{multline*}
	\end{block}
	\begin{itemize}
		\item Obtain samples $\bz$ from the encoder $q(\bz | \bx, \bphi)$.
		\item Apply flow model $\bz^* = f(\bz, \blambda)$.
		\item Compute likelihood for $\bz^*$ using the decoder, base distribution for $\bz^*$ and the Jacobian.
	\end{itemize}
	\myfootnotewithlink{https://arxiv.org/abs/1505.05770}{Rezende D. J., Mohamed S. Variational Inference with Normalizing Flows, 2015} 
\end{frame}
%=======
\begin{frame}{Inverse autoregressive flow (IAF)}
	\vspace{-0.3cm}
	\begin{align*}
		\bx &= g(\bz, \btheta) \quad \Rightarrow \quad x_i = \tilde{\sigma}_i (\bz_{1:i-1}) \cdot z_i + \tilde{\mu}_i(\bz_{1:i-1}). \\
		\bz &= f(\bx, \btheta) \quad \Rightarrow \quad z_i = \left(x_i - \tilde{\mu}_i(\bz_{1:i-1}) \right) \cdot \frac{1}{\tilde{\sigma}_i (\bz_{1:i-1})}.
	\end{align*}
	\vspace{-0.5cm}
	\begin{block}{Reverse KL for flow model}
  		\vspace{-0.5cm}
		\[
			KL(p || \pi)  = \bbE_{p(\bz)} \left[  \log p(\bz) - \log \left|\det \left( \frac{\partial g(\bz, \btheta)}{\partial \bz} \right) \right| - \log \pi(g(\bz, \btheta)) \right]
		\]
		\vspace{-0.3cm}
	\end{block}
	\begin{itemize}
	\item We don’t need to think about computing the function $f(\bx, \btheta)$.
	\item Inverse autoregressive flow is a natural choice for using flows in VAE:
	\end{itemize}
	\vspace{-0.3cm}
	\begin{align*}
		\bz &= \bsigma(\bx) \odot \bepsilon + \bmu(\bx), \quad \bepsilon \sim \mathcal{N}(0, 1); \quad  \sim q(\bz | \bx, \bphi). \\
		\bz_k &= \tilde{\bsigma}_k(\bz_{k - 1}) \odot \bz_{k-1} + \tilde{\bmu}_k(\bz_{k - 1}), \quad k \geq 1; \quad  \sim q_k(\bz_k | \bx, \bphi, \{\blambda_j\}_{j=1}^k).
	\end{align*}
	\myfootnotewithlink{https://arxiv.org/abs/1606.04934}{Kingma D. P. et al. Improving Variational Inference with Inverse Autoregressive Flow, 2016} 
\end{frame}
%=======
\begin{frame}{Inverse autoregressive flow (IAF)}
	\begin{figure}
		\includegraphics[width=\linewidth]{figs/iaf2.png}
	\end{figure}
	\begin{figure}
		\includegraphics[width=\linewidth]{figs/iaf1.png}
	\end{figure}

	\myfootnotewithlink{https://arxiv.org/abs/1606.04934}{Kingma D. P. et al. Improving Variational Inference with Inverse Autoregressive Flow, 2016} 
\end{frame}
%=======
\begin{frame}{Flows-based VAE prior vs posterior}
	\begin{block}{Theorem}
	VAE with the flow-based prior for latent code $\bz$ is equivalent to VAE with flow-based posterior for latent code $\bz$.
	\end{block}
	\begin{block}{Proof}
	\vspace{-0.5cm}
	\begin{align*}
		\mathcal{L}(\bphi, \btheta, \blambda) &= \mathbb{E}_{q(\bz | \bx, \bphi)} \log p(\bx | \bz, \btheta) - \underbrace{KL( q(\bz | \bx, \bphi) || p(\bz | \blambda))}_{\text{flow-based prior}} \\
		& = \mathbb{E}_{q(\bz | \bx, \bphi)} \log p(\bx | \bz, \btheta) - \underbrace{KL( q(\bz | \bx, \bphi, \blambda) || p(\bz))}_{\text{flow-based posterior}}
	\end{align*}
	(Here we use Flow KL duality theorem from Lecture 4)
	\end{block}
	\begin{block}{Flows in VAE posterior}
		\vspace{-0.3cm}
		{\small
		\[
			\mathcal{L} (\bphi, \btheta, \blambda) 
			= \mathbb{E}_{q(\bz | \bx, \bphi)} \bigg[\log p(\bx, \bz^* | \btheta) -  \log q(\bz | \bx, \bphi ) +   \log \left| \det \left( \frac{\partial f(\bz, \blambda)}{\partial \bz} \right) \right| \bigg].
		\]
		}
	\end{block}
	\myfootnotewithlink{https://arxiv.org/abs/1611.02731}{Chen X. et al. Variational Lossy Autoencoder, 2016}
\end{frame}
%=======
\begin{frame}{Flows-based VAE prior vs posterior}
	\begin{itemize}
		\item IAF posterior decoder path: $p(\bx|\bz, \btheta)$, $\bz \sim p(\bz)$.
		\item AF prior decoder path: $p(\bx|\bz, \btheta)$, $\bz = g(\bepsilon, \blambda)$, $\bepsilon \sim p(\bepsilon)$. 
	\end{itemize}
	The AF prior and the IAF posterior have the same computation cost, so using the AF prior makes the model more expressive at no training time cost.

	\begin{figure}
		\includegraphics[width=0.85\linewidth]{figs/prior_vs_posterior}
	\end{figure}

	\myfootnotewithlink{https://courses.cs.washington.edu/courses/cse599i/20au/slides/L09_flow.pdf}{image credit: https://courses.cs.washington.edu/courses/cse599i/20au}
\end{frame}
%=======
\begin{frame}{VAE limitations}
	\begin{itemize}
		\item Poor variational posterior distribution (encoder)
		\[
			q(\bz | \bx, \bphi) = \mathcal{N}(\bz| \bmu_{\bphi}(\bx), \bsigma^2_{\bphi}(\bx)).
		\]
		\item Poor prior distribution
		\[
			p(\bz) = \mathcal{N}(0, \mathbf{I}).
		\]
		\item Poor probabilistic model (decoder)
		\[
			p(\bx | \bz, \btheta) = \mathcal{N}(\bx| \bmu_{\btheta}(\bz), \bsigma^2_{\btheta}(\bz)).
		\]
		\item Loose lower bound
		\[
			\log p(\bx | \btheta) - \mathcal{L}(q, \btheta) = (?).
		\]
	\end{itemize}
\end{frame}
%=======
\begin{frame}{Summary}
	\begin{itemize}
		\item VampPrior proposes to use a variational mixture of posteriors as the prior to approximate the aggregated posterior.
		\vfill
		\item We could use flow-based prior in VAE (moreover, autoregressive).
		\vfill
		\item We could use flows to make variational posterior more expressive. This is equivalent to the flow-based prior. 
	\end{itemize}
\end{frame}

\end{document} 