\input{../utils/preamble}
\createdgmtitle{13}

\usepackage{tikz}

\usetikzlibrary{arrows,shapes,positioning,shadows,trees}
%--------------------------------------------------------------------------------
\begin{document}
%--------------------------------------------------------------------------------
\begin{frame}[noframenumbering,plain]
%\thispagestyle{empty}
\titlepage
\end{frame}
%=======
\begin{frame}{Outline}
	\tableofcontents
\end{frame}
%=======
\begin{frame}{Recap of previous lecture}

\end{frame}
%=======
\section{Score matching}
%=======
\begin{frame}{Score matching}
	We could sample from the model if we have $\nabla_{\bx}\log p(\bx| \btheta)$.
	\begin{block}{Fisher divergence}
		\vspace{-0.3cm}
		\[
			D_F(\pi, p) = \frac{1}{2}\bbE_{\pi}\bigl\| \nabla_{\bx}\log p(\bx| \btheta) - \nabla_\bx \log \pi(\bx) \bigr\|^2_2 \rightarrow \min_{\btheta}
		\]
		\vspace{-0.5cm}
	\end{block}
	\begin{block}{Score function}
		\vspace{-0.3cm}
		\[
			\bs(\bx, \btheta) = \nabla_{\bx}\log p(\bx| \btheta)
		\]
		\vspace{-0.3cm}
	\end{block}
	\textbf{Problem:} we do not know $\nabla_\bx \log \pi(\bx)$.
	\begin{block}{Theorem}
		Under some regularity conditions, it holds
		\vspace{-0.2cm}
		\[
			\frac{1}{2} \bbE_{\pi}\bigl\| \bs(\bx, \btheta) - \nabla_\bx \log \pi(\bx) \bigr\|^2_2 = \bbE_{\pi}\Bigr[ \frac{1}{2}\| \bs(\bx, \btheta) \|_2^2 + \text{tr}\bigl(\nabla_{\bx} \bs(\bx, \btheta)\bigr) \Bigr] + \text{const}
		\]
		\vspace{-0.4cm}
	\end{block}
	Here $\nabla_{\bx} \bs(\bx, \btheta) = \nabla_{\bx}^2 \log p(\bx | \btheta)$ is a Hessian matrix.
	\myfootnotewithlink{https://jmlr.org/papers/volume6/hyvarinen05a/old.pdf}{Hyvarinen A. Estimation of non-normalized statistical models by score matching, 2005} 
\end{frame}
%=======
\begin{frame}{Score matching}
	\begin{block}{Theorem}
		\vspace{-0.6cm}
		\[
			\frac{1}{2} \bbE_{\pi}\bigl\| \bs(\bx, \btheta) - \nabla_\bx \log \pi(\bx) \bigr\|^2_2 = \bbE_{\pi}\Bigr[ \frac{1}{2}\| \bs(\bx, \btheta) \|_2^2 + \text{tr}\bigl(\nabla_{\bx} \bs(\bx, \btheta)\bigr) \Bigr] + \text{const}
		\]
		\vspace{-0.6cm}
	\end{block}
	\begin{block}{Proof (only for 1D)}
		\vspace{-0.6cm}
		{\small
		\begin{multline*}
			\bbE_{\pi}\bigl\| s(x) - \nabla_x \log \pi(x) \bigr\|^2_2 = \bbE_{\pi} \bigl[ s(x)^2 + (\nabla_x \log \pi(x))^2 - 2{\color{teal}[s(x) \nabla_x \log \pi(x) ] \bigr] }
		\end{multline*}
		\vspace{-0.8cm}
		\begin{align*}
			{\color{teal}\bbE_{\pi} [{\color{violet}s(x) } \nabla_x \log \pi(x) ] } &= \int {\color{olive}\pi(x) {\color{violet}\nabla_x \log p(x)} \nabla_x \log \pi(x)} d x \\ 
			&= \int {\color{violet}\nabla_x \log p(x) } {\color{olive}\nabla_x \pi(x) } dx = \Bigl.\pi(x) \nabla_x \log p(x) \Bigr|_{-\infty}^{+\infty} \\
			&- \int \nabla^2_x \log p(x)  \pi(x) dx =- \bbE_{\pi} \nabla^2_x \log p(x)
		\end{align*}
		\[
			\frac{1}{2} \bbE_{\pi}\bigl\| s(x) - \nabla_x \log \pi(x) \bigr\|^2_2 = \frac{1}{2} \bbE_{\pi} \Bigr[s(x)^2 + \nabla_x s(x) \Bigl]+ \text{const}.
		\]
		}
	\end{block}
	\myfootnotewithlink{https://jmlr.org/papers/volume6/hyvarinen05a/old.pdf}{Hyvarinen A. Estimation of non-normalized statistical models by score matching, 2005} 
\end{frame}
%=======
\begin{frame}{Score matching}
	\vspace{-0.3cm}
	\begin{figure}
		\centering
		\includegraphics[width=0.8\linewidth]{figs/smld}
	\end{figure}
	\vspace{-0.6cm}
	\begin{block}{Theorem}
		\vspace{-0.6cm}
		\[
			\frac{1}{2} \bbE_{\pi}\bigl\| \bs(\bx, \btheta) - \nabla_\bx \log \pi(\bx) \bigr\|^2_2 = \bbE_{\pi}\Bigr[ \frac{1}{2}\| \bs(\bx, \btheta) \|_2^2 + \text{tr}\bigl(\nabla_{\bx} \bs(\bx, \btheta)\bigr) \Bigr] + \text{const}
		\]
		\vspace{-0.4cm}
	\end{block}
	\begin{enumerate}
	    \item The left hand side is intractable due to unknown $\pi(\bx)$ -- \textbf{denoising score matching}. 
	    \item The right hand side is complex due to Hessian matrix -- \textbf{sliced score matching}.
	\end{enumerate}
	\myfootnotewithlink{https://yang-song.github.io/blog/2019/ssm/}{Song Y. Generative Modeling by Estimating Gradients of the Data Distribution, blog post, 2021}
\end{frame}
%=======
\begin{frame}{Score matching}
	\vspace{-0.3cm}
	\begin{block}{Sliced score matching (Hutchinson's trace estimation)}
		\vspace{-0.3cm}
		\[
			\bbE_{\pi}\Bigr[ \text{tr}\bigl(\nabla_{\bx} \bs(\bx, \btheta)\bigr) \Bigr] = \bbE_{\pi(\bx)} \mathbb{E}_{p(\bepsilon)} \left[ {\color{violet}\bepsilon^T \nabla_{\bx} \bs(\bx, \btheta)} \bepsilon \right],
		\]
		where $\mathbb{E} [\bepsilon] = 0$ and $\text{Cov} (\bepsilon) = I$.
	\end{block}
	\vspace{-0.2cm}
	\begin{block}{Denoising score mathing}
		Let perturb original data by normal noise $p(\bx | \bx', \sigma) = \cN(\bx | \bx', \sigma^2 \bI)$
		\vspace{-0.3cm}
		\[
			\pi(\bx | \sigma) = \int \pi(\bx') p(\bx | \bx', \sigma) d\bx'.
		\]
		\vspace{-0.6cm} \\
		Then the solution of 
		\vspace{-0.2cm}
		\[
			\frac{1}{2} \bbE_{\pi(\bx | \sigma)}\bigl\| \bs(\bx, \btheta, \sigma) - \nabla_\bx \log \pi(\bx | \sigma) \bigr\|^2_2 \rightarrow \min_{\btheta}
		\]
		\vspace{-0.5cm} \\
		satisfies $\bs(\bx, \btheta, \sigma) \approx \bs(\bx, \btheta, 0) = \bs(\bx, \btheta)$ using small enough noise scale $\sigma$.
	\end{block}
	\myfootnote{\href{https://arxiv.org/abs/1905.07088}{Song Y. Sliced Score Matching: A Scalable Approach to Density and Score Estimation, 2019} \\
	\href{http://www.iro.umontreal.ca/~vincentp/Publications/smdae_techreport.pdf}{Vincent P. A connection between score matching and denoising autoencoders. Neural computation, 2011}}.
\end{frame}
%=======
\begin{frame}{Denoising score matching}
	\begin{block}{Theorem}
		\vspace{-0.8cm}
		\begin{multline*}
			\bbE_{\pi(\bx | \sigma)}\bigl\| \bs(\bx, \btheta, \sigma) - \nabla_\bx \log \pi(\bx | \sigma) \bigr\|^2_2 = \\ = \bbE_{\pi(\bx')} \bbE_{p(\bx | \bx', \sigma)}\bigl\| \bs(\bx, \btheta, \sigma) - \nabla_\bx \log p(\bx | \bx', \sigma) \bigr\|^2_2	
		\end{multline*}
		\vspace{-0.8cm}
	\end{block}
	\begin{figure}
		\includegraphics[width=0.75\linewidth]{figs/pitfalls}
		\includegraphics[width=0.75\linewidth]{figs/single_noise}
	\end{figure}
	\myfootnotewithlink{https://yang-song.github.io/blog/2019/ssm/}{Song Y. Generative Modeling by Estimating Gradients of the Data Distribution, blog post, 2021}
\end{frame}
%=======
\begin{frame}{Summary}
	\begin{itemize}
		\item Score matching proposes to minimize Fisher divergence to get score function.
		\vfill
		\item Sliced score matching and denoised score matching are two techniques to get scalable algorithm for fitting Fisher divergence.
	\end{itemize}
\end{frame}
\end{document} 