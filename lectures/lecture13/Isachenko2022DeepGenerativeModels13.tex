\input{../utils/preamble}
\createdgmtitle{13}
%--------------------------------------------------------------------------------
\begin{document}
%--------------------------------------------------------------------------------
\begin{frame}[noframenumbering,plain]
%\thispagestyle{empty}
\titlepage
\end{frame}
%=======
\begin{frame}{Outline}
	\tableofcontents
\end{frame}
%=======
\section{Neural ODE: finish}
%=======
\begin{frame}{Neural ODE}
	\begin{block}{Adjoint functions}
		\vspace{-0.3cm}
		\[
			\ba_{\bz}(t) = \frac{\partial L(\by)}{\partial \bz(t)}; \quad \ba_{\btheta}(t) = \frac{\partial L(\by)}{\partial \btheta(t)}.
		\]
		\vspace{-0.6cm}
	\end{block}
	\begin{block}{Theorem (Pontryagin)}
	\vspace{-0.6cm}
	\[
	     \frac{d \ba_{\bz}(t)}{dt} = - \ba_{\bz}(t)^T \cdot \frac{\partial f(\bz(t), \btheta)}{\partial \bz}; \quad \frac{d \ba_{\btheta}(t)}{dt} = - \ba_{\bz}(t)^T \cdot \frac{\partial f(\bz(t), \btheta)}{\partial \btheta}.
	\]
	Do we know any initilal condition?
	\end{block}
	\begin{block}{Solution for adjoint function}
		\vspace{-0.8cm}
		\begin{align*}
			 \frac{\partial L}{\partial \btheta(t_0)} &= \ba_{\btheta}(t_0) =  - \int_{t_1}^{t_0} \ba_{\bz}(t)^T \frac{\partial f(\bz(t), \btheta)}{\partial \btheta(t)} dt + 0\\
			 \frac{\partial L}{\partial \bz(t_0)} &= \ba_{\bz}(t_0) =  - \int_{t_1}^{t_0} \ba_{\bz}(t)^T \frac{\partial f(\bz(t), \btheta)}{\partial \bz(t)} dt + \frac{\partial L}{\partial \bz(t_1)}\\
		\end{align*}
		\vspace{-1.5cm}
	\end{block}
	\textbf{Note:} These equations are solved back in time.
	\myfootnotewithlink{https://arxiv.org/abs/1806.07366}{Chen R. T. Q. et al. Neural Ordinary Differential Equations, 2018}   
\end{frame}
%=======
\begin{frame}{Neural ODE}
	\vspace{-0.2cm}
	\begin{block}{Forward pass}
		\vspace{-0.5cm}
		\[
			\bz(t_1) = \int^{t_1}_{t_0} f(\bz(t), \btheta) d t  + \bz_0 \quad \Rightarrow \quad \text{ODE Solver}
		\]
		\vspace{-0.6cm}
	\end{block}
	\begin{block}{Backward pass}
		\vspace{-0.8cm}
		\begin{equation*}
			\left.
				{\footnotesize 
				\begin{aligned}
					\frac{\partial L}{\partial \btheta(t_0)} &= \ba_{\btheta}(t_0) =  - \int_{t_1}^{t_0} \ba_{\bz}(t)^T \frac{\partial f(\bz(t), \btheta)}{\partial \btheta(t)} dt + 0 \\
					\frac{\partial L}{\partial \bz(t_0)} &= \ba_{\bz}(t_0) =  - \int_{t_1}^{t_0} \ba_{\bz}(t)^T \frac{\partial f(\bz(t), \btheta)}{\partial \bz(t)} dt + \frac{\partial L}{\partial \bz(t_1)} \\
					\bz(t_0) &= - \int^{t_0}_{t_1} f(\bz(t), \btheta) d t  + \bz_1.
				\end{aligned}
				}
			\right\rbrace
			 \Rightarrow
			\text{ODE Solver}
		\end{equation*}
		\vspace{-0.4cm} 
	\end{block}
	\textbf{Note:} These scary formulas are the standard backprop in the discrete case.
	\begin{figure}
		\centering
		\includegraphics[width=\linewidth]{figs/neural_ode}
	\end{figure}
	\myfootnotewithlink{https://arxiv.org/abs/1806.07366}{Chen R. T. Q. et al. Neural Ordinary Differential Equations, 2018}   
\end{frame}
%=======
\section{Continuous-in-time normalizing flows}
%=======
\begin{frame}{Continuous Normalizing Flows}
	\begin{block}{Discrete Normalizing Flows}
		\vspace{-0.8cm}
		  \[
		  \bz_{t+1} = f(\bz_t, \btheta); \quad \log p(\bz_{t+1}) = \log p(\bz_{t}) - \log \left| \det \frac{\partial f(\bz_t, \btheta)}{\partial \bz_{t}} \right| .
		  \]
		\vspace{-0.7cm}
	\end{block}
	\begin{block}{Continuous-in-time dynamic transformation}
		\vspace{-0.2cm}
		\[
			\frac{d\bz(t)}{dt} = f(\bz(t), \btheta).
		\]
		\vspace{-0.4cm}
	\end{block}
	Assume that function $f$ is uniformly Lipschitz continuous in $\bz$ and continuous in $t$. From Picard’s existence theorem, it follows that the above ODE has a \textbf{unique solution}.
	\begin{block}{Forward and inverse transforms}
		\vspace{-0.7cm}
		\begin{align*}
			\bx &= \bz(t_1) = \bz(t_0) + \int_{t_0}^{t_1} f(\bz(t), \btheta) dt \\
			\bz &= \bz(t_0) = \bz(t_1) + \int_{t_1}^{t_0} f(\bz(t), \btheta) dt \\
		\end{align*}
	\end{block}
	\myfootnotewithlink{https://arxiv.org/abs/1912.02762}{Papamakarios G. et al. Normalizing flows for probabilistic modeling and inference, 2019}   
\end{frame}
%=======
\begin{frame}{Continuous Normalizing Flows}
	To train this flow we have to get the way to calculate the density~$p(\bz(t))$.
	\begin{block}{Theorem (Fokker-Planck)}
		if function $f$ is uniformly Lipschitz continuous in $\bz$ and continuous in $t$, then
		\vspace{-0.3cm}
		\[
			\frac{\partial \log p(\bz(t))}{\partial t} = - \text{trace} \left( \frac{\partial f (\bz(t), \btheta)}{\partial \bz(t)} \right).
		\]
		\vspace{-0.5cm}
	\end{block}
	\textbf{Note:} Unlike discrete-in-time flows, the function $f$ does not need to be bijective, because uniqueness guarantees that the entire transformation is automatically bijective.
	\begin{block}{Density evaluation}
		\vspace{-0.4cm}
		\[
			\log p(\bx | \btheta) = \log p(\bz) - \int_{t_0}^{t_1} \text{trace}  \left( \frac{\partial f (\bz(t), \btheta)}{\partial \bz(t)} \right) dt.
		\]
		\textbf{Adjoint} method is used to integral evaluation.
	\end{block}
	\myfootnotewithlink{https://arxiv.org/abs/1806.07366}{Chen R. T. Q. et al. Neural Ordinary Differential Equations, 2018}   
\end{frame}
%=======
\begin{frame}{Continuous Normalizing Flows}
	\vspace{-0.6cm}
	\begin{block}{Forward transform + log-density}
		\vspace{-0.8cm}
		\[
			\begin{bmatrix}
				\bx \\
				\log p(\bx | \btheta)
			\end{bmatrix}
			= 
			\begin{bmatrix}
				\bz \\
				\log p(\bz)
			\end{bmatrix} + 
			\int_{t_0}^{t_1} 
			\begin{bmatrix}
				f(\bz(t), \btheta) \\
				- \text{trace} \left( \frac{\partial f(\bz(t), \btheta)}{\partial \bz(t)} \right) 
			\end{bmatrix} dt.
		\]
		\vspace{-0.6cm}
	\end{block}
	\begin{itemize}
		\item Discrete-in-time normalizing flows need invertible $f$. It costs $O(d^3)$ to get determinant of Jacobian.
		\item Continuous-in-time flows require only smoothness of $f$. It costs $O(d^2)$ to get trace of Jacobian.
	\end{itemize}
	\vspace{-0.5cm}
	\begin{minipage}[t]{0.4\columnwidth}
		\begin{figure}
			\centering
			\includegraphics[width=0.75\linewidth]{figs/cnf_flow.png}
		\end{figure}
	\end{minipage}%
	\begin{minipage}[t]{0.6\columnwidth}
		\begin{figure}
			  \centering
			  \includegraphics[width=0.8\linewidth]{figs/ffjord.png}
		\end{figure}
	\end{minipage}
	\myfootnotewithlink{https://arxiv.org/abs/1810.01367}{Grathwohl W. et al. FFJORD: Free-form Continuous Dynamics for Scalable Reversible Generative Models, 2018} 
\end{frame}
%=======
\section{Diffusion Models}
%=======
\begin{frame}{Summary}
	\begin{itemize}
		\item Adjoint method generalizes backpropagation procedure and allows to train Neural ODE solving ODE for adjoint function back in time.
		\vfill
		\item Fokker-Planck theorem allows to construct continuous-in-time normalizing flow with less functional restrictions.
		\vfill
		\item FFJORD model makes such kind of flows scalable.
		\vfill
	\end{itemize}
\end{frame}
\end{document} 